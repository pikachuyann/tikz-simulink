% simulink.tex --- Documentation (and examples) for the LaTeX package.
%% Copyright 2016-2017 Yann Duplouy
%
% Permission is hereby granted, free of charge, to any person obtaining a copy
% of this software and associated documentation files (the "Software"), to deal
% in the Software without restriction, including without limitation the rights
% to use, copy, modify, merge, publish, distribute, sublicense, and/or sell
% copies of the Software, and to permit persons to whom the Software is
% furnished to do so, subject to the following conditions:
%
% The above copyright notice and this permission notice shall be included in
% all copies or substantial portions of the Software.
%
% THE SOFTWARE IS PROVIDED "AS IS", WITHOUT WARRANTY OF ANY KIND, EXPRESS OR
% IMPLIED, INCLUDING BUT NOT LIMITED TO THE WARRANTIES OF MERCHANTABILITY,
% FITNESS FOR A PARTICULAR PURPOSE AND NONINFRINGEMENT. IN NO EVENT SHALL THE
% AUTHORS OR COPYRIGHT HOLDERS BE LIABLE FOR ANY CLAIM, DAMAGES OR OTHER
% LIABILITY, WHETHER IN AN ACTION OF CONTRACT, TORT OR OTHERWISE, ARISING FROM,
% OUT OF OR IN CONNECTION WITH THE SOFTWARE OR THE USE OR OTHER DEALINGS IN THE
% SOFTWARE.
%
%%%%%%%%%%%%%%%%%%%%%%%%%%%%%%%%%%% LaTeX Preamble %%%%%%%%%%%%%%%%%%%%%%%%%%%%
\documentclass[a4paper]{article}

\usepackage{a4wide}
\usepackage{hyperref}
\usepackage{moreverb}
\usepackage{simulink}
\usepackage{caption}

%%%%%%%%%%%%%%%%%%%%%%%%%%%%%%%%%%% Document References %%%%%%%%%%%%%%%%%%%%%%%
\newcommand{\skstyversion}{$\alpha$ release}
\title{simulink.sty --- A \LaTeX{} package to easily create Simulink models
       with TikZ.\\ {\large \skstyversion}}
\author{(c) 2016--2017 Yann Duplouy}

%%%%%%%%%%%%%%%%%%%%%%%%%%%%%%%%%%%%%%%%%%%%%%%%%%%%%%%%%%%%%%%%%%%%%%%%%%%%%%%
\begin{document}
    \maketitle
    \vspace{-0.5cm}

    \tableofcontents
    \clearpage

\section{Introduction}
    {\tt simulink.sty} is a \LaTeX{} package that defines new TikZ (or more
exactly, PGF) shapes that depict various Simulink blocks. This package should
allow to draw more precisely Simulink models, and replace the current way of
drawing Simulink models, which is by taking screenshots of Matlab-Simulink.

\section{Currently available Simulink Blocks}
    The following Simulink blocks are currently available in the package :
\begin{center}
    \begin{tabular}{|c|c|c|c|}
        \hline
        {\bf Block Name} & {\bf Shape Name} & {\bf Default shape} & 
        {\bf Better Shape} \\
        \hline \hline
        Inport & {\tt skInport} & \skBlockO{skInport}{$\ell_1$} & \\
        \hline
        Outport & {\tt skOutport} & \skBlockI{skOutport}{$\ell_1$} & \\
        \hline
        Constant & {\tt skConstant} &
        \skBlockO{skConstant,cstval=$c$}{$\ell_1$} & \\
        \hline
        Add (2-ports) & {\tt skAdd} &
        \skBlockIIO{skAdd}{$\ell_1$}{$\ell_2$}{$\ell_3$} & \\
        \hline
        Unit Delay & {\tt skUnitDelay} &
        \skBlockIO{skUnitDelay}{$\ell_1$}{$\ell_2$} & \\
        \hline
        Delay & {\tt skDelay} &
        \skBlockIO{skDelay,delayval=k}{$\ell_1$}{$\ell_2$} & \\
        \hline
        Zero-Order Hold & {\tt skZeroOrderHold} &
        \skBlockIO{skZeroOrderHold}{$\ell_1$}{$\ell_2$} & \\
        \hline
        Gain & {\tt skGain} & \skBlockIO{skGain}{$\ell_1$}{$\ell_2$} & \\
        \hline
        Integrator & {\tt skIntegrator} &
        \skBlockIO{skIntegrator}{$\ell_1$}{$\ell_2$} & 
        \SKBettertrue
        \skBlockIO{skIntegrator}{$\ell_1$}{$\ell_2$}
        \SKBetterfalse \\
        \hline
        Sine Wave (continuous) & {\tt skSineWave} &
        \skBlockO{skSineWave}{$\ell_1$} & \\
        \hline
        Transport Delay & {\tt skTransportDelay} &
        \skBlockIO{skTransportDelay}{$\ell_1$}{$\ell_2$} & \\
        \hline
        Relay & {\tt skRelay} & \skBlockIO{skRelay}{$\ell_1$}{$\ell_2$} & \\
        \hline
        Switch & {\tt skSwitch} &
        \skBlockIIIO{skSwitch}{$\ell_1$}{$\ell_2$}{$\ell_3$}{$\ell_4$} & \\
        \hline
        Scope & {\tt skScope} & \skBlockI{skScope}{$\ell_1$} & \\
        \hline
    \end{tabular}
\end{center}

    Some of the blocks have shapes that are difficult to recognize for someone
    not used to Simulink. For that matter, an option, named
    \texttt{BetterShape}, has been added to \texttt{simulink.sty} so that
    blocks display in a more understandable shape.

\section{An example of Simulink model drawn through TikZ}
\begin{center}
    \begin{tikzpicture}
    \node[skConstant,draw,cstval=1] at (0,0.1cm) (b1) {};
    \node[skAdd,draw,block height=1.2cm] at (5cm,-0.2cm) (b2) {};
    \node[skUnitDelay,draw] at (2cm,-0.5cm) (b3) {};
    \node[skOutport,draw] at (8cm,-0.2cm) (b4) {};
    \path[->] (b3.portB) edge node[below] {$\ell_2$} (b2.portB);
    \path[->] (b1.portA) edge node[above] {$\ell_1$} (b2.portA);
    \path[->] (b2.portC) edge node[above] {$\ell_3$} (b4.portA);
    \coordinate[right=1cm of b2.portC] (c3);
    \coordinate[below=1.2cm of c3] (c4);
    \coordinate[left=1cm of b3.portA] (c5);
    \coordinate[below=0.9cm of c5] (c6);
    \path (b2.portC) edge (c3); \path (c3) edge (c4);
    \path (c4) edge (c6); \path (c6) edge (c5);
    \path[->] (c5) edge (b3.portA);
\end{tikzpicture}

    \captionof{figure}{A small Simulink model} \label{DelayAndAdd}
\end{center}
    One might draw the Simulink model shown in figure \ref{DelayAndAdd} by
writing the following \LaTeX{} code :
\verbatiminput{Figures/Delay_and_Add.tex}
\end{document}
